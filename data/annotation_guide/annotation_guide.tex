\documentclass{article}
\usepackage[english]{babel}
\usepackage{tabularx}

\title{Annotation Guideline}

\author{Jorge del Pozo, Nicolas Obregon Royo, Kamil Kojs, Julius Roder \\
        \small IT University Copenhagen
}
\date{}

\begin{document}
\maketitle

\section{Overview}
\paragraph{Context}
The goal of our project is to look for any correlation between the diversity of a movie cast and the general sentiment towards that movie or the actors and actresses. The data we are aiming to annotate is tweets referencing a selection of movies based on hashtags.

\paragraph{Annotation Guideline}
This is the annotation guideline we are using for manual annotation of the sentiment of these movies.
\begin{table}[hbt!]
    \centering
    \begin{tabularx}\textwidth{l|X}
        Label & Annotation Guideline \\ \hline
        Negative & A generally negative sentiment towards the movie, the cast or a single member of the cast. \\ \hline
        Negative-racial & A generally negative sentiment towards the movie, the cast or a single member of the cast, based on the race of a cast member. \\ \hline
        Neutral & Either a tweet that is not referencing the movie or a generally neutral sentiment towards the movie, the cast or a single member of the cast. \\ \hline
        Positive & A generally positive sentiment towards the movie, the cast or a single member of the cast. \\ \hline
        Positive-racial & A generally positive sentiment towards the movie, the cast or a single member of the cast, based on the race of a cast member. \\ 
        
    \end{tabularx}
    \label{tab:my_label}
\end{table}
\end{document}